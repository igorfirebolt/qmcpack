\chapter{Conversion and visualization utilites for QMCPACK}\label{conversion}
QMCPACK is not a fully standalone simulation system, but resides in an
ecosystem of software, including packages to generate
pseudopotentials, electronic structure codes to generate orbitals,
etc.  A number of tools have been developed to facilitate transfering
data into and out of the file formats used by QMCPACK.

\section{Pseudopotential conversion}
Although psuedopotential file formats are generally fairly simple, and
some standards have been proposed, every major electronic structure
code has adopted its own proprietary format for essentially the same
data.  To overcome this problem, we have a developed a 

\section{Orbital conversion}
For typical materials and molecules simulations, QMCPACK requires
the single-particle orbitals as input.  At present, for orbitals in
Gaussian bases, the bases and coefficients are input directly in
QMCPACK's XML format.  Alternatively, the orbitals may be represented
either on a real-space mesh or a plane-wave expansion.  In these
latter cases, the information is stored in a cross-platform binary
format known as HDF5.  In particular, we have established a set of
conventions for storing the atomic geometry, orbitals, and related
information in and HDF5 in a format called ES-HDF.  In addition, we
have included tools to convert to ES-HDF from the output of a number
of common electronic structure codes:

\begin{table}[H]
\centering
\begin{tabular}{|c|c|c|c|c|}
\hline
Source format            & source basis & tool &
dest. basis & output format \\\hline
ABINIT                   & PW  & wfconv       & RSM or MB & ES-HDF \\
CASINO                   & PW  & wfconv       & RSM or MB & ES-HDF \\
GAMESS                   & G   & wfconv       & RSM       & ES-HDF \\
Quantum Espresso (PWscf) & PW  & pw2qmcpack.x & RSM or MB & ES-HDF \\
Qbox                     & RSM & wfconv       & RSM or MB & ES-HDF \\
GAUSSIAN                 & G   & convert4qmc  & G   & XML    \\
\hline
\end{tabular}
\caption{Formats for converting orbitals for QMCPACK.  PW=''plane
  waves'', G=''gaussians'', RSM=''real-space mesh'', and MB=''mixed basis''.}
\end{table}

\subsection{Using wfconv}
\subsubsection{ABINIT}
\texttt{istwfk 1}
\subsubsection{CASINO}
\subsubsection{GAMESS}
\subsubsection{Qbox}
\subsection{Using the mixed-basis representation}

\subsection{Converting from PWscf}


\section{Visualizing orbitals with wfvis}
\subsection{Building}
\subsection{Basic operations}
\subsection{Ray-tracing for publication-quality results}

\subsection{ES-HDF file format}

ES-HDF
\begin{itemize}
  \item format {\color{red} string}  M  ``ES-HDF''
  \item version {\color{red} int[3]} M Major, Minor,Patch
  \item application {\color{red} group} 
  \item supercell   {\color{red} group}
  \begin{itemize}
    \item primitive\_vectors {\color{red} double[3][3]}  M primitive vectors
    \item boundary\_conditions  {\color{red} int[3]      }  M  1 for periodic, 0 for open
  \end{itemize}
\end{itemize}
%version   int   [3]   M   Major,Minor,Patch
%schema_url   string   1   M   URL for ES-HDF schema
%application
%Something on who, when, why, what and how
%supercell
%primitive_vectors   double   [3][3]   M   primitive vectors
%boundary_conditions   int   [3]   O   1 for periodic, 0 for open boundary
%atoms
%number_of_atoms   int   1   M   
%number_of_species   int   1   M   
%species_ids   int   [number_of_atoms]   M   
%positions   double   [number_of_atoms][3]   O   
%reduced_positions   double   [number_of_atoms][3]   M   
%forces   double   [number_of_atoms][3]   O   
%species_0
%name   string   1   M   
%valence_charge   int   1   M   
%atomic_number   int   1   O   
%pseudopotential   string   1   O   
%mass   double   1   O   
%species_1
%electrons
%number_of_electrons   int   2   M   
%number_of_kpoints   int   1   M   
%functional   string   1   M   
%total_energy   double   1   M   
%number_of_spins   int   1   M   
%5psi_r_is_complex   bool   1   M   
%psi_r_mesh   int   [3]   M   [0,L) excluding the end
%density
%number_of_gvectors   int   1   M   
%gvectors   int   [number_of_gvectors][3]   M   
%mesh   int   [3]   M   [0,L) excluding the end
%spin_0
%density_r   double   [n0][n1][n2]   M   
%density_g   double   [number_of_gvectors][2]   M   
%spin_1
%kpoint_0
%reduced_k   double   [3]   M   reduced unit, maybe reduced_k is better (JK)
%weight   double   1   M   
%number_of_gvectors   int   1   M   
%gvectors   int   [number_of_gvectors][3]   M   
%spin_0
%number_of_states   int   1   M   
%eigenvalues   double   [number_of_states]   M   
%occupations   double   [number_of_states]   M   
%state_0
%psi_g   double   [number_of_gvectors][2]   O    
%psi_r   double   [n0][n1][n2](if real) or [n0][n1][n2][2] (if complex)   O   
%muffin_tin_0
%tin_num   int   1       
%u_lm_r   double   [num_r][(lmax+1)^2][2]       
%du_lm_final   double   [(lmax+1)^2][2]       
%muffin_tin_1
%...
%state_1
%spin_1
%...
%...
%kpoint_1
%...
%muffin_tins
%number_of_tins   int   1   O   
%muffin_tin_0
%pos   double   3   {{{4}}}   {{{5}}}
%lmax   int   1   {{{4}}}   {{{5}}}
%num_radial_points   int   1   {{{4}}}   {{{5}}}
%radial_points   int   [num_radial_points]   {{{4}}}   {{{5}}}
%muffin_tin_1
%...

