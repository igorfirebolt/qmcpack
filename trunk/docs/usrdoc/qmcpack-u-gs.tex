\chapter{Getting Started}
QMCPACK is written in C++. It is designed for massively parallel computers and
optimized for multi/many-core architectures.  It employs MPI/OpenMP hybrid
programming model for parallel executions. QMCPACK is being actively developed
by and for the QMC community and therefore this document may not reflect the most
updated status of QMCPACK and may contain inaccurate information.
Consult http://qmcpack.cmscc.org (wiki) and  http://qmcpack.googlecode.com/ (code
repository)  for current information including contact information, bug reports, etc. 
%developer contact info here (Jeongnim? or all of us?)
\section{Obtaining QMCPACK}
%QMCPACK is written in C++ and is currently under the New BSD License.  Due to
%its incomplete compliance with the current ISO C++ standard, it is not released
%in cross-compilable packages.  It is instead obtained from its source code
%repository hosted by Google Code. http://code.google.com/p/qmcpack/, 
The latest revision of the code can
be checked out anonymously using Subversion, with the command
\begin{term}
svn checkout http://qmcpack.googlecode.com/svn/trunk/ qmcpack-read-only
\end{term}
where the local directory name \icode{qmcpack-read-only} can be changed as
needed.  The repository forbids anonymous users (ie. nonmembers of the project)
from committing changes to the code.  The above command will create a directory
\icode{qmcpack-read-only} (topdir) which contains
\begin{verbatim} 
qmcpack-read-only/
  CMakeLists.txt
  src/           : directory for the source
    CMakeLists.txt
    dir1/
      CMakeLists.txt
    dir2/
      CMakeLists.txt
    ....
  CMake/        : directory with cmake files for external package and compiler
  config/       : directory with toolchain files for tested systems
  docs/         : directory with Doxygen files
  utils/        : directory with utility scripts
  build/        : empty directory for a build 
\end{verbatim}

%can pretty much copy the svn instructions from google code
\section{Building QMCPACK}
%The aim of this section is to provide a guideline for building QMCPACK in the
%most generic way.  A compromise must be made between overly many case-by-case
%examples and an oversimplified procedure.

%Theoretically speaking, 
QMCPACK can be used on any *nix machine with C++ compilers that are reasonably
recent.  Some examples to date are g++ $\geq 4.2$ (GNU) and icc $\geq 10.1$
(Intel).  It uses cmake to build internal libraries and QMC applications.
There are other required packages which the compilation process will
point out if missing.  Listed below are those required specifically by QMCPACK,
with the oldest tested versions.
\begin{itemize*}
\item{} CMake $\geq 2.8$
\item{} MPI library
\item{} blas and lapack library
\item{} boost library
\item{} fftw library $\geq 3$
\item{} xml2 library
\item{} hdf5 library $\geq 1.6$
%\item{} pygtkglext: OpenGl extensions for Python GTK bindings for the GUI interface.
% This package has some dependencies.
\end{itemize*}
For library packages, their full development versions (with suffixes -dev or
-devel) are needed.  No additional environment variables need to be set if the
above packages are installed from precompiled binaries.\footnote{Typical
examples are rpm or deb packages.}  If instead the libraries are locally built,
their installation directories must be explicitly set as environment variables,
eg. \icode{LIBXML2\_HOME=/usr/local}.  See the installation notes in each
package for details.  On many HPC centers, these packages are managed by
utilities like \icode{module} and \icode{softenv}.  Again, see the
documentation on each site for details.
% recent [meaning $> 4.2$ at least] GNU
%Compiling the QMCPACK source requires the CMake package.
%For cmake documentation and guides, consult cmake wiki.

Assuming that all of the above packages are properly set up, building QMCPACK proceeds as follows.  Change to qmcpack the top directory (name can be different).
\begin{term}
cd qmcpack-read-only
\end{term}
We recommend out-of-source compilation by creating a directory for the libraries and binaries that is separate from the source directory.  If not in \iterm{qmcpack-read-only} already, create the build directory (eg. \iterm{build}) then change to the directory.\footnote{Like the top directory, the build directory name is completely arbitrary and it can be created at any location, even outside \iterm{qmcpack-read-only}.  This is useful if you need more than one build, for testing purposes.}
\begin{term}
mkdir build
cd build
\end{term}
In the build directory, run cmake to create Makefiles, then build the executable using make.
%(Don't forget ..)
\begin{term}
cmake ..
make
\end{term}
If everything goes well, then you should see \iterm{qmcpack-read-only/build/bin/qmcapp}.
%The procedure above, creating build directory and running camke in a new directory, is an example. We can further separate the source (development) and build. Let's assume that the QMCPACK topdir is /home/foo/src/qmcpack. Then, one can build multiple executables in different locations by creating new directories and build QMCPACK in each directory.
%\begin{verbatim}
%/home/foo/build/gcc-real
%/home/foo/build/gcc-complex
%/home/foo/build/mpi-real
%....
%\end{verbatim}
%
%In each directory, e.g., /home/foo/build/gcc-real (after setting the environments properly), execute
%\begin{verbatim}
%$cmake /home/foo/src/qmcpack
%$make 
%\end{verbatim}
So far, there is no need to change sources or cmake files.  \icode{cmake ..} in the above procedure uses \iterm{..} because the source tree resides in the parent directory.  If something did not work, simply remove the directory (eg. \icode{rm -rf build}) and start again. 

Additional configurations must be considered if you need to deviate from the default settings of parallel computing (MPI) and multithreading (OpenMP).  See \S{}\ref{ss:compset} for details.

%discuss toolchains here
\subsection{Compiler settings for MPI and OpenMP} \label{ss:compset}
%To run QMCPACK under MPI, 
\paragraph{MPI}
Running QMCPACK under MPI requires it to be built with a modified compiler.  This usually involves installing an additional \emph{compiler wrapper} package which turns the already working compiler into an MPI-capable one, without having to rebuild the compiler itself.\footnote{MPI implementations of this type are known as \emph{portable implementations}.}  Examples of MPI implementations known to work with QMCPACK are OpenMPI, MPICH2, and MVAPICH.
%icc vs. g++

Some environment variables must be set in order to use MPI.  By default, the variable CXX is set to the serial compiler (either g++ or icc).  Change this to one of the following:
%MPI is automatically enabled if
\begin{itemize*}
  \item{} mpicxx, mpic++, cmpic++ (tungsten at NCSA)
  \item{} mpCC/mpCC\_r on AIX 
\end{itemize*}
Next, add the directory containing mpi.h to the include file search path.  If the compiler wrappers were installed from binary packages or were locally built with default options, they should already be in one of the directories in the standard search path, eg. /usr/include or /usr/local/include.  If you want to keep MPI enabled for other projects but disable it for QMCPACK only, do one of the following:
%s, eg., /usr/include or /usr/local/include
%  \begin{itemize*}
%    \item{} SGI Altix  [?]
%  \end{itemize*}
%One of these actions will disable MPI
\begin{itemize*}
  \item{} Modify \iterm{topdir/CMakeLists.txt}:
\begin{code}
SET(QMC_MPI 0)
\end{code}
  \item{} Set \icode{QMC\_MPI} environment to 0, eg. \icode{export QMC\_MPI=0} for bash.
\end{itemize*}
\textbf{Note}: Shell environment variables take precedence over CMake settings.

\paragraph{OpenMP}
Multithreading with OpenMP is automatically enabled if CMake detects that a compiler
support OpenMP. These compilers have been extensively tested for OpenMP:
\begin{itemize*}
  \item{} GNU/OpenMP compilers $>$ 4.2.x on Linux 2.6.x kernels, Mac OS X 
  \item{} Intel compilers
  \item{} IBM VisualAge compilers
\end{itemize*}
To enable OpenMP for other compilers, one of the following actions will be needed:
\begin{itemize*}
  \item{} Modify \iterm{topdir/CMakeLists.txt}:
\begin{code}
SET(QMC_OMP 1)
\end{code}
  \item{} Set \icode{QMC\_OMP} to 1, eg. \icode{export QMC\_OMP=1} for bash.
\end{itemize*}

If your machine has multiple cores, there is no need to disable OpenMP.
Note that the default number of threads on your machine may be set to the
number of cores (or CPU units). It is always safe to set the number of threads
yourself as, for example, 
\begin{code}
export OMP_NUM_THREADS=1
\end{code}
It is important to set the  environment variables which control underlying
libraries that can utilize OpenMP or any other threading methods.  Especially
with MKL, set
\begin{code}
MKL_NUM_THREADS=1
MKL_SERIAL=YES
\end{code}
so that the blas/lapack calls DO NOT USE threaded version.

%More on cmake
%cmake environment variables
In addition to \icode{QMC\_MPI} and \icode{QMC\_OMP}, there are few more
environment variables used by QMCPACK and CMake to determine compiler-time
options. \icode{QMC\_COMPLEX} (0 or 1) sets wavefunctions to take real or
complex values.\footnote{One can use real wavefunction for the complex-enabled
build but it will be extremely inefficient (4-8 times slower). However, complex
wavefunctions cannot be used with real-enabled build.}  \icode{QMC\_BITS} (32
or 64) sets the OS bit size.  A change in any one of these variables will cause
Make to rebuild everything instead of rebuilding just the modified source
files.
%Each build should use identical variables. If the working shell has different variables from the previous build environments, cmake/make will rebuild everything which can take time. Out-of-source compilation becomes very useful to build different combinations.
%Note that separate executables have to be built for real and complex wavefunctions.  

%\paragraph{External libraries}
%%If lapack/blas or atlas is not in your standard path, do one of the following. ``location'' is where the libraries are located.
%If LAPACK/BLAS or ATLAS is not in your library search path, set the environment
%variables \icode{LAPACK} to \icode{"-L/installdir -llapack -lblas"} and
%\icode{ATLAS} to\\ \icode{"-L/installdir -llapack -lf77blas -lcblas -latlas"},
%changing \iterm{installdir} appropriately.

%For bash users,
%\begin{verbatim}
% export LAPACK="-L/location -llapack -lblas"
% export ATLAS="-L/location -llapack -lf77blas -lcblas -latlas" 
%\end{verbatim}
%For tcsh users,
%\begin{verbatim}
% setenv LAPACK "-L/location -llapack -lblas"
% setenv ATLAS "-L/location -llapack -lf77blas -lcblas -latlas"
%\end{verbatim}
%%JNKIM COMMENTED OUT: too tedious
%%\subsection{Configurations known to work}
%%\subsubsection{Intel 64 Abe cluster @ NCSA}
%%Abe uses SoftEnv to activate the configurations for the (otherwise dormant) software installed in it.  Create the file .soft in the home directory, containing the following lines:\footnote{These settings were checked and confirmed to work on July 22, 2010.  NCSA unfortunately is an ever-experimental cluster which frequently installs new software and deletes old ones without notice.  If you find that these settings no longer work, contact \href{mailto:dcyang2@illinois.edu}{ChangMo Yang}.}
%%\begin{term}
%%@remove +cmake-2.6.3
%%@remove +intel-10.1.017
%%@remove +hdf5-1.6.7
%%@remove +phdf5-1.6.7
%%+cmake-2.8.1
%%+intel-11.1.072
%%+intel-mkl-10.2.2
%%+hdf5-1.8.4
%%+fftw-3.1-intel
%%+libxml2
%%@default
%%@teragrid-basic
%%\end{term}
%%Also needed are the packages \href{http://www.boost.org/}{Boost} and \href{http://einspline.sourceforge.net/}{Einspline}.  They are to be installed locally, ie. usually under your home directory \iterm{\$\{HOME\}}.  Refer to their websites for further download and installation instructions.  After they are correctly installed set the following environment variables, using either the \iterm{export} or the \iterm{setenv} command depending on the shell:
%%\begin{term}
%%CXX=mpicxx
%%CC=mpicc
%%LIBXML2_HOME=/usr/local/libxml-2.6.32
%%BOOST_HOME=\$\{HOME\}/include
%%EINSPLINE_HOME=\$\{HOME\}
%%HDF5_API=16
%%QMC_BITS=64
%%QMC_MPI=1
%%QMC_OMP=1
%%\end{term}
%%
%%\subsubsection{Intel Mac OS X}
%%There are two Linux-style package management systems for Intel-based Mac computers: \href{http://www.macports.org/}{MacPorts} and \href{http://www.finkproject.org/}{Fink}.  We only list the packages and environment settings that are needed to compile QMCPACK.\footnote{Building from the source is perfectly acceptable.}  Note that once CMake is installed, it will use the Mac Framework and link the libraries automatically.
%%\paragraph{MacPorts}
%%\subparagraph{Ports} cmake, gcc44, atlas, boost, svn, hdf5-18, libxml2, fftw-3, zlib
%%\subparagraph{Environment variables}
%%\begin{term}
%%HDF5_API=16
%%\end{term}
%%
%%\paragraph{Fink}
%%\subparagraph{Packages} cmake, hdf5, libxml2, svn, boost
%%\begin{term}
%%HDF5_HOME=/sw
%%LIBXML2_HOME=/sw
%%\end{term}
%%%For the Boost package, set \iterm{BOOST\_HOME} to the directory where the package was decompressed.  There is no compiling involved.
%%%[edit] Few facts
%%%    * Do not install lapack/blas via fink or port
%%%[edit] Extras
%%%    * The performance of plane-wave depends on blas library. Framework or GotoBlas do not perform as well as MKL.
%%%    * If Intel compilers and MKL are available, set the environments as 
%%%\begin{verbatim}
%%%export CXX=icpc
%%%export CC=icc
%%%export MKL_HOME=/Library/Frameworks/Intel_MKL.framework/Versions/Current
%%%\end{verbatim}
%%%    * Then, cmake/make
%%Also check \iterm{DYLD\_LIBRARY\_PATH}. Add the lines below (replace \iterm{32} with \iterm{64} for 64 bit OS) in the shell startup.
%%
%%\begin{term}
%%if [ -z "\$\{LD_LIBRARY_PATH\}" ]
%%then
%%  LD_LIBRARY_PATH=\$MKL_HOME/lib/32; export LD_LIBRARY_
%%PATH
%%else
%%  LD_LIBRARY_PATH=\$MKL_HOME/lib/32:\$LD_LIBRARY_PATH; export LD_LIBRARY_PATH
%%fi
%%
%%if [ -z "\$\{DYLD_LIBRARY_PATH\}" ]
%%then
%%  DYLD_LIBRARY_PATH=\$MKL_HOME/lib/32; export DYLD_LIBRARY_PATH
%%else
%%  DYLD_LIBRARY_PATH=\$MKL_HOME/lib/32:\$DYLD_LIBRARY_PATH; export DYLD_LIBRARY_PATH
%%fi
%%\end{term}

\subsection{Tested environments}
QMCPACK is developed for large-scale high-performance computing systems. We update the status of QMCPACK on the HPC systems the developers have access to and share experiences in dealing with some quirkiness of each system.

In general, the quickest way to build and use QMCPACK is to use a toolchain
file for each system. They are available in \iterm{config} directory with the
distribution. We recommend using cmake 2.8.x or higher on each HPC system. They are usually
available via \icode{module} or \icode{softenv}. 
The current list includes
\begin{itemize*}
\item{} AbeMvapich2.20091104.cmake : abe@ncsa
\item{} AbeMvapich\_CUDA.cmake : abe@ncsa, enable CUDA
\item{} JaguarGNU.cmake : jaguar@ornl, Cray XT
\item{} KrakenGNU.cmake : kraken@nics, Cray XT
\item{} Longhorn.cmake : longhorn@tacc
\item{} Longhorn\_CUDA.cmake : longhorn@tacc, enable CUDA
\item{} LinuxIntel.cmake : generic for LINUX using Intel compilers
\end{itemize*}
From the top directory of QMCPACK, we can configure the build environment
according to the chosen toolchain (eg. \iterm{mychain.cmake}) as follows.
\begin{term}
cd build
cmake -DCMAKE_TOOLCHAIN_FILE=../config/mychain.cmake ..
cmake -DCMAKE_TOOLCHAIN_FILE=../config/mychain.cmake ..
make
\end{term}
%Feel free to make a local copy in build directory and modify it for your environment.
% Where the toolchain file is located is not important.
If you choose to modify any of the existing toolchain files for your
environment, only the paths to the external libraries have to be modified for
each platform. 

%\subsubsection{Cygwin [clean this]}
%Update qmcpack using svn, especially src/CMakeLists.txt.
%\paragraph{Summary}
%\begin{itemize}
%\item{} cygwin 1.5.25-7
%\item{} GCC 3.4.4 
%\end{itemize}
%\paragraph{Libraries}
%\begin{itemize}
%\item{} cmake
%\item{} libxml2 : Set \iterm{LIBXML2\_HOME=/usr/include/libxml2}
%\item{} boost : select boost-devel using cygwin update.
%\begin{itemize}
%\item{} Set BOOST\_HOME=/usr/include/boost-1\_33\_1
%\item{} Or search the directory where boost/config.hpp resides 
%\end{itemize}
%\item{} HDF5
%\begin{itemize}
%\item{} Download source version 1.6.x (not 1.8.x) from the HDF5 home page.
%\item{} After installation, set \iterm{HDF5\_HOME=/usr/local}
%\end{itemize}
%\end{itemize}

\subsection{Known version conflicts}
None considered so far.

\section{A walk-through example with Si}
I guess Jaron knows more about this...
\subsection{Input file layout}
mention just the standard-ness of xml and postpone further discussion to next chapter
\subsection{Preparing wavefunctions}
\subsubsection{Wavefunction optimization}
\subsection{Your first VMC calculation}
\subsection{Your first DMC calculation}
\subsection{Understanding the output}
mention data analysis tool and postpone discussion for two chapters
%Obtaining QMCPACK
%  SVN instructions
%Building QMCPACK
%  Simple build instructions
%  Building with toolchain files (machine specific)
%  Building QMCPACK from scratch
%    Library Dependencies and where to obtain them
%    Detailed build instructions
%  Using the benchmark database to test your build
%  Known Good Configurations (combinations of library versions that work)
%  Known version conflicts
%  Who to contact if you are experiencing difficulties
